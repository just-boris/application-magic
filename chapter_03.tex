\chapter{Моделирование}
\section{Инструменты моделирования}

По выраженным выше формулам построим графики. Для расчетов используем бибилотеку SciPy для языка программирования Python. В этой библиотеке есть все необходимые нам математические функции, а также возможность скомпоновать из них более сложные для наших расчетов.

Для графической части используем пакет Mathplotlib, гибкий и мощный построитель графиков. 

\section{Моделирование поперечного смещения}

При поперечном смещении центр выходного поля смещается относительно входного на расстояние $\delta x$. В ходе моделироывания, получились следующие зависимости:

\begin{figure}[ht!]
	\begin{subfigure}{.48\textwidth}
		\includegraphics[width=1\linewidth]{}
		\caption{Зависимость коэффициента передачи}
	\end{subfigure}
	\hfill
	\begin{subfigure}{.48\textwidth}
		\includegraphics[width=1\linewidth]{}
		\caption{Схематичный вид}
	\end{subfigure}
	\caption{Поперечное смещение}
\end{figure}