\chapter{Безопасность жизнедеятельности}

Целью данной дипломной работы является разработка программы, для расчета эффективности связи мод двух волноводов. Разработка производилась на ЭВМ. В связи с этим, в данной главе рассмотрены вопросы организации рабочего места и безопасности работы с ЭВМ. Дипломная работа выполнялась в исследовательской лаборатории, расположенной на территории кафедры ФиТОС университета СПбГУ~ИТМО, поэтому, в главе освещены вопросы электро- и пожаробезопасности на предприятиях, а также рассмотрены опасные и вредные производственные факторы.

\section{Вредные и опасные факторы при работе на персональных электронно-вычислительных машинах}

Основным фактором, влияющим на производительность труда людей, работающих с ПЭВМ, являются комфортные и безопасные условия труда. Условия труда пользователя, работающего с персональным компьютером, определяются:
\begin{itemize}
	\item особенностями организации рабочего места;
	\item условиями производственной среды (освещением, микроклиматом, шумом, электромагнитными и 			\item электростатическими полями, визуальными эргономическими параметрами дисплея и т. д.);
	\item характеристиками информационного взаимодействия человека и персональных электронно-вычислительных машин.
\end{itemize}

Согласно ГОСТу 12.0.003-74 "ССБТ. Опасные и вредные производственные факторы. Классификация" существуют несколько групп опасных и вредных факторов, сопутствующих работе с ПЭВМ.

\noindent Физические:
\begin{itemize}
	\item повышенная или пониженная температура воздуха рабочей зоны;
	\item выделение в воздух рабочей зоны ряда химических веществ;
	\item повышенная или пониженная влажность воздуха;
	\item повышенный или пониженный уровень отрицательных и положительных аэроионов;
	\item повышенное значение напряжения в электрической цепи, замыкание;
	\item повышенный уровень статического электричества;
	\item повышенный уровень электромагнитных излучений;
	\item повышенная напряженность электрического поля;
	\item отсутствие или недостаток естественного света;
	\item недостаточная искусственная освещенность рабочей зоны;
	\item повышенная яркость света;
	\item повышенная контрастность;
\end{itemize}

\noindent Психофизологические:
\begin{itemize}
	\item монотонность трудового процесса;
	\item нервно-эмоциональные перегрузки.
	\item умственное перенапряжение
	\item перенапряжение зрительного анализатора
	\item длительные статические нагрузки
\end{itemize}

Работа на ПК сопровождается постоянным и значительным напряжением функций зрительного анализатора. Одной из основных особенностей является иной принцип чтения информации, чем при обычном чтении. При обычном чтении текст на бумаге, расположенный горизонтально на столе, считывается работником с наклоненной головой при падении светового потока на текст. При работе на ПК оператор считывает текст, почти не наклоняя голову, глаза смотрят прямо или почти прямо вперед, текст (источник — люминесцирующее вещество экрана) формируется по другую сторону экрана, поэтому пользователь не считывает отраженный текст, а смотрит непосредственно на источник света, что вынуждает глаза и орган зрения в целом работать в несвойственном ему стрессовом режиме длительное время.

Расстройство органов зрения резко увеличивается при работе более четырех часов в день. Нервно-эмоциональное напряжение при работе на ПК возникает вследствие дефицита времени, большого объема и плотности информации, особенностей диалогового режима общения человека и ПК, ответственности за безошибочность информации.
Для существенного уменьшения боли и неприятных ощущений, возникающих у пользователей ПК, необходимы частые перерывы в работе и эргономические усовершенствования, в том числе оборудование рабочего места так, чтобы исключать неудобные позы и длительные напряжения.

\section{Обеспечение безопасности труда при работе с ПЭВМ}
Согласно СанПиН 2.2.2/2.4.1340-03 "Гигиенические требования к электронно-вычислительным машинам и организации работы", основными требованиями при организации работы на ПЭВМ являются \cite{sanpin}:
\begin{itemize}
	\item Требования к ПЭВМ
	\item Требования к помещениям для работы с ПЭВМ
	\item Требования к микроклимату, содержанию аэроионов и вредных химических  веществ в воздухе на рабочих местах, оборудованных ПЭВМ
	\item Требования к уровням шума на рабочих местах, оборудованных ПЭВМ	
	\item Требования к освещению на рабочих местах, оборудованных ПЭВМ
	\item Требования к уровням электромагнитных полей на рабочих местах,  оборудованных ПЭВМ
	\item Общие требования к организации рабочих мест пользователей ПЭВМ
	\item Требования к организации и оборудованию рабочих мест с ПЭВМ для  взрослых пользователей
\end{itemize}

\subsection{Требования к помещениям и обеспечивающие их мероприятия}
Согласно санитарным правилам и нормам помещения для работы с ПЭВМ помещения для эксплуатации ПЭВМ должны  иметь естественное и искусственное освещение. Эксплуатация ПЭВМ в помещениях без естественного освещения допускается только при соответствующем  обосновании  и  наличии положительного  санитарно-эпидемиологического  заключения,    выданного в установленном порядке. Окна в помещениях,  где эксплуатируется  вычислительная  техника,  преимущественно  должны   быть ориентированы на север и северо-восток. Оконные проемы должны быть  оборудованы  регулируемыми  устройствами типа: жалюзи, занавесей, внешних козырьков и др.

Площадь на одно рабочее место пользователей ПЭВМ на  базе электроннолучевой трубки  (ЭЛТ)  должна  составлять  не  менее 6 $\textsc{м}^2$. При использовании ПЭВМ  на  базе  ЭЛТ, отвечающих требованиям  международных стандартов безопасности компьютеров, с  продолжительностью  работы  менее 4-х часов в день допускается минимальная площадь 4.5 $\textsc{м}^2$ на  одно  рабочее место  пользователя.

Источники света, такие как светильники и окна, которые дают отражение от поверхности экрана, значительно ухудшают точность знаков и влекут за собой помехи физиологического характера, которые могут выразиться в значительном напряжении, особенно при продолжительной работе. Отражение, включая отражения от вторичных источников света, должно быть сведено к минимуму. Для защиты от избыточной яркости окон могут быть применены шторы и экраны.

\subsection{Требования к микроклимату в помещении и средства его обеспечения}
В производственных помещениях, в которых работа с использованием ПЭВМ является основной и связана с нервно-эмоциональным напряжением, предлагаются оптимальные параметры микроклимата, представленные в таблице \ref{climate}


\begin{table}[h]
\caption{Микроклиматические требования к воздуху рабочей зоны}
\label{climate}
\begin{tabular}{|c|c|c|}
\hline
	Требование & Оптимальное & Допустимое \\
\hline
	Температура воздуха & $22^\circ \textsc{C}$ & $22-24^\circ \textsc{C}$ \\
\hline
	Относительная влажность & 40-60\% & не более 75\% \\
\hline
	Скорость движения воздуха & не более 0.1 м/с & не более 0.1 м/с\\
\hline
\end{tabular}
\end{table}

Концентрации вредных веществ, выделяемых ПЭВМ в воздух помещений не превышают предельно допустимых концентраций (ПДК), установленных для атмосферного воздуха. В помещениях, оборудованных ПЭВМ, проводится ежедневная влажная уборка и систематическое проветривание после каждого часа работы на ПЭВМ.

\subsection{Требования к освещению и его устройство в помещении с ПЭВМ}
Работа, выполняемая с использованием вычислительной техники, имеют следующие недостатки:
\begin{itemize}
	\item вероятность появления прямой блесткости;  
	\item ухудшенная контрастность между изображением и фоном;
	\item отражение экрана.
\end{itemize}

Рабочее помещение должно иметь естественное освещение, коэффициент естественной освещенности должен быть не менее 1.2-1.5\%, согласно СНиП 23-05-95 «Естественное и искусственное освещение».
Освещенность на поверхности стола в зоне размещения рабочего документа должна быть 300-500 лк. Следует ограничивать отраженную блесткость на рабочих поверхностях (экран, стол, клавиатура и др.) за счет правильного выбора типов светильников и расположения рабочих мест по отношению к источникам естественного и искусственного освещения, при этом яркость бликов на экране ПЭВМ не должна превышать 40 кд/м2 и яркость потолка не должна превышать 200 $\textsc{кд/м}^2$ согласно требованию СанПин 2.2.2/2.4.1340-03.

Показатель ослепленности для источников общего искусственного освещения в производственных помещениях должен быть не более 20. Показатель дискомфорта для административно-общественных помещений - не более 40. 

\subsection{Требования к уровню шума и методы его снижения и мероприятия для их обеспечения}
Согласно ГОСТ 12.1.003-76 "Система стандартов безопасности труда. Шум. Общие требования безопасности" уровень шума на рабочем месте программистов и операторов ПК не должен превышать 50 дБА, а в залах обработки информации на вычислительных машинах – 65 дБА. Для того чтобы добиться этого уровня  шума рекомендуется применять звукопоглощающее покрытие стен. 
В качестве мер по снижению шума можно предложить следующее:
\begin{itemize}
	\item облицовка потолка и стен звукопоглощающим материалом (снижает шум на 6-8 дБ);
	\item экранирование рабочего места (постановкой перегородок, диафрагм);
	\item установка оборудования, производящего минимальный шум, в отдельных компьютерных помещениях;
	\item рациональная планировка помещения.
\end{itemize}
 
Защиту от шума следует выполнять в соответствии с ГОСТ 12.1.003-76, а звукоизоляция ограждающих конструкций должна отвечать требованиям главы СНиП 11-12-77 «Защита от шума. Нормы проектирования». 

\subsection{Требования к уровню неионизирующих и ионизирующих излучений, меры защиты}
Данное требование предусматривает рассмотрение двух видов излучения: неионизирующее (тепловое действие, электромагнитная волна) и ионизирующее (рентгеновское излучение). 

Согласно СанПиН 2.2.2/2.4.1340-03 максимальный уровень рентгеновского излучения на рабочем месте оператора компьютера обычно не превышает 10 мкбэр/ч, а интенсивность ультрафиолетового и инфракрасного излучений от экрана монитора лежит в пределах 10–100 $\textsc{мВт/м}^2$. Уровни напряженности электростатических полей должны составлять не более 20 кВ/м. Поверхностный электростатический потенциал не должен превышать 500В.

Для снижения воздействия перечисленных видов излучения на операторов ПК рекомендуется применять мониторы с пониженной излучательной способностью, устанавливать защитные экраны, а также соблюдать регламентированные режимы труда и отдыха, делать пятнадцатиминутные перерывы в течении полутора часов работы.

\subsection{Требования к организации и оборудованию рабочих мест с ПЭВМ}
Проектирование рабочих мест, снабженных видеотерминалами, относится к числу важных проблем эргономического проектирования в области вычислительной техники. Располагать все элементы на рабочем столе следует соответственно антропометрическим, физическим и психологическим требованиям. Большое значение имеет также характер работы. В частности, при организации рабочего места программиста необходимо соблюсти следующие условия: оптимально разместить оборудование, входящее в состав рабочего места и организовать достаточное рабочее пространство, позволяющее осуществлять все необходимые движения и перемещения.

Главными элементами рабочего места программиста являются стол и кресло. Основным рабочим положением является положение сидя. Рабочее место пользователя должно занимать площадь не менее 6 $\textsc{м}^2$, высота помещения должна быть не менее  4 м, а объем - не менее 20 м$\textsc{м}^3$ на одного человека.

Для комфортной работы необходимо, чтобы стол удовлетворял следующим условиям:
\begin{itemize}
	\item высота стола была выбрана с учетом возможности сидеть свободно, в удобной позе, при необходимости опираясь на подлокотники;
	\item нижняя часть стола сконструирована так, чтобы программист мог удобно сидеть, не был вынужден поджимать ноги;
	\item поверхность часть стола должна обладала свойствами, исключающими появление бликов в поле зрения программиста; 
\end{itemize}

Высота над уровнем пола рабочей  поверхности, за которой работает пользователь, должна составлять 720 мм. Рабочий стол пользователя при необходимости должен регулироваться по высоте в пределах 680 - 780 мм. Оптимальные  размеры  поверхности  стола 1600 х 1000 кв. мм. Под столом должно иметься пространство для ног с размерами по глубине 650 мм. Рабочий стол пользователя должен также иметь подставку для ног, расположенную под углом $15^\circ$ к поверхности стола. Длина подставки 400 мм, ширина - 350 мм. Расстояние, на которое удалена клавиатура от края стола, должна быть не более 300 мм, что обеспечит пользователю удобную опору для предплечий. Расстояние между глазами оператора и экраном видеодисплея должно составлять 60-70 см.

\subsection{Требования к организации труда и отдыха пользователей ПЭВМ и обеспечивающие их мероприятия}
Согласно СанПиН 2.2.2.542-96 режимы труда и отдыха при работе с ПЭВМ должны организовываться в зависимости от вида и категории трудовой деятельности.

Так как работа пользователя ПЭВМ по виду трудовой деятельности относится к группе В – творческая работа в режиме диалога с ЭВМ, а по напряженности работы ко II категории тяжести (суммарное число считываемых или вводимых знаков за рабочую смену, не более 40 000 знаков за смену), то суммарное время перерывов должно составлять 50 минут за 8-ми часовую смену.

Кроме того, во время регламентированных перерывов с целью снижения нервно-эмоционального напряжения, утомления зрительного анализатора, предотвращения развития утомления целесообразно выполнять комплексы физических упражнений.

\subsection{Требования электробезопасности в помещении с ПЭВМ и обеспечивающие их мероприятия}
Помещение лаборатории по опасности поражения электрическим током можно отнести к 1 классу, т.е. это помещение без повышенной опасности (сухое,  бес пыльное,  с нормальной температурой воздуха, изолированными полами и малым числом заземленных приборов).

В течении работы на корпусе компьютера накапливается статическое электричество. На расстоянии 5-10 см от экрана напряженность электростатического поля составляет 60-280 кВ/м, то есть в 10 раз  превышает  норму  20 кВ/м. Для уменьшения напряжённости применять применение увлажнители и нейтрализаторы, антистатическое покрытия пола.

Основным организационным мероприятием является инструктаж и обучение безопасным методам труда, а так же проверка знаний правил безопасности и инструкций в соответствии с занимаемой должностью применительно к выполняемой работе.
При эксплуатации ПЭВМ должны быть соблюдены  следующие требования электробезопасности:
\begin{itemize}
	\item сетевое электропитание  устройств ПЭВМ должно производиться только от розеток типа "Европа" с заземляющими контактами;
	\item все электрические розетки, предназначенные для подключения к ним устройств ПЭВМ, должны иметь маркировку по напряжению. Значение  номинального напряжения сети (220 В) необходимо наносить яркой краской, крупными символами (высотой не менее 50 мм) на стене или щите, возле или над розеткой
	\item заземляющие контакты розеток должны иметь  соединения с заземляющим контуром помещения или должны быть занулены. При занулении необходимо обратить особое внимание на создание надежного контакта нулевого провода с нулевой шиной сети электропитания;
\end{itemize}

\section{Пожаробезопасность}

Основными причинами возникновения пожара являются:
\begin{itemize}
	\item нарушение установленных правил пожарной безопасности и неосторожное обращение с огнём;
	\item неисправность и перегрузка электрических устройств (короткое замыкание);
	\item халатное и неосторожное обращение с огнём;
	\item статическое электричество, образующееся от трения пыли или газов в вентиляционных установках;
	\item грозовые разряды при отсутствии или неисправности молниеотводов.
\end{itemize}
При работе в исследовательско-производственной  лаборатории, предъявляются следующие требования обеспечения пожаробезопасности:
\begin{itemize}
	\item Вход в помещение, проходы между столами и коридоры не разрешается загромождать различными предметами и оборудованием. Для хранения всех веществ и материалов предусматриваем специальные шкафы и ёмкости.
	\item С рабочими и обслуживающим персоналом предусматривается проведение противопожарного инструктажа, занятий и бесед.
\end{itemize}
Размещение средств пожаротушения и их маркировка должны соответствовать ГОСТ 12.1.004-76.