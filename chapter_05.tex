\chapter{Безопасность жизнедеятельности}

\section{Введение}
Целью данной дипломной работы является разработка программы, для расчета эффективности связи мод двух волноводов. Разработка производилась на ЭВМ. В связи с этим, в данной главе рассмотрены вопросы организации рабочего места и безопасности работы с ЭВМ. Дипломная работа выполнялась в исследовательской лаборатории, расположенной на территории кафедры ФиТОС университета СПбГУ~ИТМО, поэтому, в главе освещены вопросы электро- и пожаробезопасности на предприятиях, а также рассмотрены опасные и вредные производственные факторы.

\section{Вредные и опасные факторы при работе на персональных электронно-вычислительных машинах}

Основным фактором, влияющим на производительность труда людей, работающих с ПЭВМ и ВДТ, являются комфортные и безопасные условия труда. Условия труда пользователя, работающего с персональным компьютером, определяются:
\begin{itemize}
	\item особенностями организации рабочего места;
	\item условиями производственной среды (освещением, микроклиматом, шумом, электромагнитными и 			\item электростатическими полями, визуальными эргономическими параметрами дисплея и т. д.);
	\item характеристиками информационного взаимодействия человека и персональных электронно-вычислительных машин.
\end{itemize}

При выполнении работ на персональном компьютере (ПК) согласно ГОСТу 12.0.003-74 "ССБТ. Опасные и вредные производственные факторы. Классификация" могут иметь место следующие факторы:
\begin{itemize}
	\item повышенная температура поверхностей ПК;
	\item повышенная или пониженная температура воздуха рабочей зоны;
	\item выделение в воздух рабочей зоны ряда химических веществ;
	\item повышенная или пониженная влажность воздуха;
	\item повышенный или пониженный уровень отрицательных и положительных аэроионов;
	\item повышенное значение напряжения в электрической цепи, замыкание;
	\item повышенный уровень статического электричества;
	\item повышенный уровень электромагнитных излучений;
	\item повышенная напряженность электрического поля;
	\item отсутствие или недостаток естественного света;
	\item недостаточная искусственная освещенность рабочей зоны;
	\item повышенная яркость света;
	\item повышенная контрастность;
	\item зрительное напряжение;
	\item монотонность трудового процесса;
	\item нервно-эмоциональные перегрузки.
\end{itemize}
Работа на ПК сопровождается постоянным и значительным напряжением функций зрительного анализатора. Одной из основных особенностей является иной принцип чтения информации, чем при обычном чтении. При обычном чтении текст на бумаге, расположенный горизонтально на столе, считывается работником с наклоненной головой при падении светового потока на текст. При работе на ПК оператор считывает текст, почти не наклоняя голову, глаза смотрят прямо или почти прямо вперед, текст (источник — люминесцирующее вещество экрана) формируется по другую сторону экрана, поэтому пользователь не считывает отраженный текст, а смотрит непосредственно на источник света, что вынуждает глаза и орган зрения в целом работать в несвойственном ему стрессовом режиме длительное время.

Расстройство органов зрения резко увеличивается при работе более четырех часов в день. Нервно-эмоциональное напряжение при работе на ПК возникает вследствие дефицита времени, большого объема и плотности информации, особенностей диалогового режима общения человека и ПК, ответственности за безошибочность информации.

Для существенного уменьшения боли и неприятных ощущений, возникающих у пользователей ПК, необходимы частые перерывы в работе и эргономические усовершенствования, в том числе оборудование рабочего места так, чтобы исключать неудобные позы и длительные напряжения.

Помещения должны иметь естественное и искусственное освещение. Площадь на одно рабочее место с компьютером для взрослых пользователей должна составлять не менее $6~\textsc{м}^2$, а объем не менее $20~\textsc{м}^3$. Помещения с компьютерами должны оборудоваться системами отопления, кондиционирования воздуха или эффективной приточно-вытяжной вентиляцией. Для внутренней отделки интерьера помещений с компьютерами должны использоваться диффузно-отражающие материалы с коэффициентом отражения для потолка — 0,7-0,8; для стен — 0,5-0,6; для пола — 0,3-0,5.
Поверхность пола в помещениях эксплуатации компьютеров должна быть ровной, без выбоин, нескользкой, удобной для очистки и влажной уборки, обладать антистатическими свойствами.

К числу факторов, ухудшающих состояние здоровья пользователей компьютерной техники, относятся электромагнитное и электростатическое поля, акустический шум, изменение ионного состава воздуха и параметров микроклимата в помещении. Немаловажную роль играют эргономические параметры расположения экрана монитора (дисплея), состояние освещенности на рабочем месте, параметры мебели и характеристики помещения, где расположена компьютерная техника.

Уровни шума на рабочих местах пользователей персональных компьютеров не должны превышать значений, установленных СанПиН 2.2.4/2.1.8.562-96 и составляют не более 50 дБА. На рабочих местах в помещениях для размещения шумных агрегатов уровень шума не должен превышать 75 дБА, а уровень вибрации в помещениях допустимых значений по СН 2.2.4/2.1.8.566-96 категория 3, тип «в».
Снизить уровень шума в помещениях можно использованием звукопоглощающих материалов с максимальными коэффициентами звукопоглощения в области частот 63-8000 Гц для отделки стен и потолка помещений. Дополнительный звукопоглощающий эффект создают однотонные занавески из плотной ткани, повешенные в складку на расстоянии 15-20 см от ограждения. Ширина занавески должна быть в 2 раза больше ширины окна.

\section{Пожаробезопасность}

Основными причинами возникновения пожара являются:
\begin{itemize}
	\item нарушение установленных правил пожарной безопасности и неосторожное обращение с огнём;
	\item неисправность и перегрузка электрических устройств (короткое замыкание);
	\item халатное и неосторожное обращение с огнём;
	\item статическое электричество, образующееся от трения пыли или газов в вентиляционных установках;
	\item грозовые разряды при отсутствии или неисправности молниеотводов.
\end{itemize}
При работе в исследовательско-производственной  лаборатории, предъявляются следующие требования обеспечения пожаробезопасности:
\begin{itemize}
	\item Вход в помещение, проходы между столами и коридоры не разрешается загромождать различными предметами и оборудованием. Для хранения всех веществ и материалов предусматриваем специальные шкафы и ёмкости.
	\item С рабочими и обслуживающим персоналом предусматривается проведение противопожарного инструктажа, занятий и бесед.
\end{itemize}
Размещение средств пожаротушения и их маркировка должны соответствовать ГОСТ 12.1.004-76.

\section{Электробезопасность}
\subsection{Характеристика возможных опасных и вредных производственных факторов}

Электробезопасность — система организационных и технических мероприятий и средств, обеспечивающих защиту людей от вредного и опасного воздействия электрического тока. Всё электрическое оборудование оптической лаборатории, такое как лазерные установки, компьютеры, безпылевые камеры и др., нуждаются в средствах электробезопасности.

ГОСТ 12.1.038-82 содержит данные предельно допустимых напряжений прикосновения и токов \cite{gost_electro}. 
Остановке сердца при поражении предшествует так называемое фибрилляционное состояние. Фибрилляция сердца заключается в беспорядочном сокращении и расслаблении мышечных волокон (фибрилл) сердца. Электрический ток, вызывающий такое состояние, называется пороговым фибрилляционным током. При переменном токе он находится в пределах 100 мА — 5 А, при постоянном токе — 300 мА — 5 А. При токе более 5 А происходит немедленная остановка сердца, минуя состояние фибрилляции. Если через сердце пострадавшего пропустить кратковременно (доли секунды) ток 4—5 А, мышцы сердца сокращаются и после отключения тока сердце продолжает работать. При остановке и фибрилляции сердца работа его самостоятельно не восстанавливается, поэтому необходимо оказание первой (доврачебной) помощи в виде искусственного дыхания и непрямого массажа сердца. В состоянии клинической смерти человек может находиться в течение 3—5 мин. Если за данный промежуток времени человеку не оказывается помощь, клиническая (мнимая) смерть переходит в биологическую (истинную) смерть — необратимый процесс отмирания клеток.

\subsection{Технические средства защиты, обеспечивающие безопасность работ}

Электрозащитные средства по назначению подразделяются на: изолирующие; ограждающие; вспомогательные. Изолирующие служат для изоляции человека от токоведущих частей и в свою очередь подразделяются на основные и дополнительные.

Основные — это те средства защиты, изоляция которых длительно выдерживает рабочее напряжение. Они позволяют прикасаться к токоведущим частям под напряжением. К ним относятся:
\begin{itemize}
	\item изолирующие штанги;
	\item изолирующие и электроизмерительные клещи;
	\item диэлектрические перчатки;
	\item диэлектрическая обувь;
	\item слесарно-монтажный инструмент с изолирующими рукоятками;
	\item указатели напряжения.
\end{itemize}
Основными мерами защиты от поражения электрическим током являются:
\begin{itemize}
	\item обеспечение недоступности токоведущих частей, находящихся под напряжением, для случайного прикосновения;
	\item электрическое разделение сети;
	\item устранение опасности поражения при появлении напряжения на корпусах, кожухах и других частях электрооборудования, что достигается защитным заземлением, занулением, защитным отключением (ГОСТ 12.1.030-81 «Электробезопасность. Защитное заземление. Зануление» ).
	\item применение малых напряжений;
	\item защита от случайного прикосновения к токоведущим частям применением кожухов, ограждений, двойной изоляции;
	\item защита от опасности при переходе напряжения с высшей стороны на низшую;
	\item контроль и профилактика повреждений изоляции;
	\item компенсация емкостной составляющей тока замыкания на землю;
	\item применение специальных электрозащитных средств — переносных приборов и предохранительных приспособлений;
	\item организация безопасной эксплуатации электроустановок.
\end{itemize}

\section{Опасные и вредные производственные факторы}
К опасным относят факторы, воздействие которых на работающих в определенных условиях приводит к травме или другому внезапному ухудшению здоровья. 

Вредными называют производственные факторы, воздействие которых на работающих, в определенных условиях приводит к заболеванию или снижению работоспособности.

\subsection{Повышенная температура, влажность и подвижность воздуха рабочей зоны}

Аппаратура системы управления является источником такого ОВПФ, как повышенная температура воздуха рабочей зоны. Этот фактор возникает из-за нагрева воздуха блоками питания, процессорами, другими электронными компонентами оборудования. Особенностью теплового режима является неравномерность распределения температур по рабочим зонам. Повышенная температура воздуха и неравномерность её распределения в рабочей зоне системы управления в основном сопровождается повышенной влажностью и скоростью движения воздуха.

Повышенная температура, относительная влажность и скорость движения воздуха приводят к повышенной утомляемости, снижению внимания, повышенному потоотделению, общему снижению производительности труда, что может привести к несчастному случаю. 

ГОСТ 12.1.005-88 устанавливает нормы температуры, относительной влажности и скорости движения воздуха в рабочей зоне производственных помещений. Для защиты от указанных факторов должны предусматриваться средства нормализации воздушной среды производственных помещений – вентиляция воздуха, кондиционирование.

\subsection{Повышенный уровень шума на рабочем месте}

Под шумом понимают звуки, мешающие восприятию полезных звуков или нарушающие тишину, а также звуки, оказывающие вредное или раздражающее действие на организм человека. Многие производственные процессы сопровождаются значительным шумом. Чрезмерный шум на производстве, уровень которого не соответствует существующим санитарным нормам, оказывает вредное влияние на организм человека: развивает тугоухость и глухоту, расшатывает центральную нервную систему, вызывает головные боли и бессонницу, учащается пульс и дыхание, изменяется кровяное давление. 

Шум является причиной более быстрого, чем в нормальных условиях, утомления и снижения работоспособности человека. Источниками шума могут быть энергетические блоки оборудования, вентиляторы охлаждения, не полностью исправное оборудование. 

ГОСТ 12.1.003-83 определяет допустимые уровни звукового давления в октавных полосах частот, уровни звука и эквивалентные уровни звука на рабочих местах. Снижение шумовых излучений производится устранением источника шума или снижением его интенсивности. Это достигается путем применения новой техники, рациональным планированием помещений. Возможно снижения шума путем его частичного устранения на пути распространения. Этот метод предполагает использование различных экранов, материалов для обшивки, глушителей и др.

\subsection{Недостаточная освещенность рабочей зоны}
Источниками фактора недостаточной освещенности являются неправильное размещение рабочих мест или осветительных приборов, недостаточная мощность источников света. Освещение, соответствующее санитарным нормам, является главнейшим условием гигиены труда и культуры производства. При хорошем освещении устраняется напряжение зрения, ускоряется темп работы. При недостаточном освещении глаза сильно напрягаются, темп работы снижается, утомляемость работников увеличивается, качество работы снижается. Недостаточное освещение рабочих мест отрицательно влияет на хрусталик глаза, что может привести к близорукости. Чрезмерно яркое освещение раздражает сетчатую оболочку глаза, вызывает ослепленность. Глаза работников сильно устают, зрительное восприятие ухудшается, растет производственный травматизм, производительность труда падает. При хорошо организованном, рациональном освещении, соответствующем санитарным нормам, эти недостатки устраняются.
СНиП 23-05-95 устанавливает нормы естественного, искусственного и совмещенного освещения зданий и сооружений. В зависимости от разряда и подразряда зрительной работы нормируются следующие показатели:
\begin{itemize}
	\item освещенность – Е, люкс, при системах общего или комбинированного освещения; 
	\item показатель ослепленности;
	\item коэффициент пульсации;
	\item КЕО, \%, при естественном и совмещенном освещении.
\end{itemize}
Для местного освещения рабочих мест следует использовать светильники с непросвечивающими отражателями. Светильники должны располагаться таким образом, чтобы их светящие элементы не попадали в поле зрения работающих на освещаемом рабочем месте и на других рабочих местах. 

Освещенность должна быть достаточной для быстрого и легкого различения объектов работы, соответствовать характеру производственных функций; не меняться во времени; быть равномерной, без резких теней; между объектом рассмотрения и фоном, на котором рассматривается объект, необходима некоторая контрастность; источник света не должен создавать бликов на объекте рассмотрения и ослеплять работающего.

Наиболее приемлемым для помещения являются люминесцентные лампы ЛБ (лампа белого  света), ЛТБ (лампа тепло-белого   света). Мощность ламп может  быть 20, 40, 80 Вт.

\subsection{Нервно-психические перегрузки}

Нервно-психические перегрузки подразделяются на: умственное перенапряжение, перенапряжение анализаторов, монотонность труда, эмоциональные перегрузки. Эти факторы производства возникают вследствие напряженной, порой даже сверхнапряженной деятельности человека, при большой ответственности и насыщенности техникой и людьми. Усугубление таких перегрузок происходит вследствие несоблюдения норм на опасные и вредные производственные факторы.

\section{Заключение}

В исследовательско-производственной лаборатории кафедры ФиТОС, приняты все меры по соблюдению правил безопасности труда и работы с электрическим оборудованием. Имеются средства индивидуальной и коллективной защиты работников, аптечка первой медицинской помощи. Помещение оснащено локальными и общим источниками света, вентиляцией. 

Лаборатория оснащена защитным заземлением. Крепления заземляющей клеммы и проводников зафиксированы от случайного развинчивания. Место присоединения заземляющего проводника обозначено нестираемым при эксплуатации знаком заземления. Вокруг клеммы заземления контактная площадка для присоединения проводника. Площадка защищена от коррозии или изготавливаться из антикоррозионного материала и не должна иметь поверхностной окраски.

Вход посторонним в исследовательско-производственную лабораторию строго воспрещен.
