\chapter*{Заключение}
\addcontentsline{toc}{chapter}{Заключение}

В результате проведенной выпускной квалификационной работы был разработан метод расчета эффективности связи двух волноводов с помощью интеграла перекрытия. Были смоделированы поля оптического волокна и полоскового волновода, а затем проведены расчеты эффективности связи при различных вариантах стыковки оптических волноводов. В итоге можно сделать следующие выводы:

\begin{itemize}
	\item При стыковке волноводов лучше не допускать смещения центра волокна относительно точки максимума передачи, которую можно определить при помощи созданной модели, более чем на 2.5 мкм, так как при этом уровень потерь на эффективности связи составляет более 1~дБ;
	\item Следует не допускать зазоров между канальным волноводом и стыкуемым волокном, поскольку расходимость светового пучка на выходе из волокна снижает эффективность передачи;
	\item Для снижения уровня обратных отражений следует создать на торце волновода скос и стыковать волокно плоскостью скоса вплотную к полосковому волноводу. Моделирование показало, что при скосе в $18^\circ$ достигается наибольшая эффективность по соотношению "эффективность стыковки - минимальные обратные отражения", однако уже при $10^\circ$ можно получить 99\% максимально возможной величины.
\end{itemize}

Результаты моделирования справедливы для конкретного волокна ОВССП 125-2-0.00002 ТУ ЯЕИЛ.48-2008, характеристики которого использовались при моделировании в работе, и для определенных параметров канала в полосковом волноводе.

Однако полученная модель универсальна, поскольку позволяет изменить указанные характеристики и пересчитать полученные значения для новых условий, поэтому рекомендуется дальнейшее использование данных средств и методов для моделирования и расчета эффективности связи двух волноводов с помощью интеграла перекрытия.