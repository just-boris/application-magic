\chapter*{Введение}
\addcontentsline{toc}{chapter}{Введение}
Использование оптических технологий и элементов является большим достижением в наше время. Меньшая длина волны позволяет достичь большей частоты сигналов, в частности при передаче данных. Такая частота дает возможность передавать потоки информации в несколько терабит в секунду. Важными преимуществами ВОЛС являются такие факторы, как малое затухание сигналов, позволяющее, при использовании современных технологий, строить участки оптических систем в сто и более километров без ретрансляции, высокая помехозащищенность, связанная с малой восприимчивостью оптического волокна к электромагнитным помехам, и многие другие.

Развитие оптической связи создало новый класс устройств - интегрально-оптические схемы и компоненты. Их устройство похоже на обычные интегральные платы и представляет собой подложку из электрооптического кристалла и выполненных в ней канальных волноводов, которые могут служить базой для изготовления различных функциональных элементов  (поляризаторов, разветвителей, модуляторов и др.). Основным достоинством интегрально-оптических устройств является их высокое быстродействие. Уже созданы интегрально-оптические  переключатели с временем переключения менее 100 фс. Такое быстродействие недостижимо для устройств обычной полупроводниковой электроники. Возможность передачи и обработки больших объемов информации определяет бурное развитие интегральной оптики в настоящее время.

Область исследований постепенно расширяется и сейчас она включает в себя все исследования, направленные на использование волноводной технологии для создания новых или усовершенствования существующих оптических приборов. Разработаны компактные и миниатюрные элементы, чей малый размер обеспечивает большую надежность, лучшую механическую и температурную стабильность, уменьшение потребления энергии и управляющих напряжений в активных приборах. Кроме того, отдельные волноводные элементы могут объединяться вместе в более сложные схемы на общей подложке или на отдельных платах. Такие новые волноводные приборы, например, лазеры и модуляторы, могут успешно конкурировать по своим индивидуальным качествам с их объемными оптическим эквивалентами.

Элементы интегрально-оптических схем схожи с элементами электронных интегральных схем и при их изготовлении используются аналогичные тех\-нологии. Это обстоятельство и обусловило возникновение самого названия новой области - "интегральная оптика". В действительности исследования по интегральной оптике были начаты около 40 лет назад. Появление лазеров стимулировало эти исследования, и были достигнуты значительные успехи. На сегодняшний день созданы все элементарные компоненты интегрально-оптических схем: планарные и канальные волноводы, распределенные устройства ввода-вывода излучения в волновод, пленочные переключатели, ответвители, модуляторы, источники излучения и фотодетекторы, а также линзы, призмы, отражатели и поляризаторы в интегральном исполнении.

Несмотря на указанные достижения, по-прежнему остается ряд проблем, затрудняющих развитие этой области. Одна из них - это проблема соединения компонентов. Интегрально-оптические схемы представляют собой подложку со сформированным в ней полосковым волноводом. Средой передачи света между фотоприемником, излучателем и другими интегральными элементами схемы служит оптическое волокно. На соединении волокна и полоскового волновода возникают оптические потери в связи с различными параметрами в волноводного  распространения света в них. Кроме того, необходимо точное согласование оптических каналов и надежная фиксация их положения относительно друг друга, поскольку небольшое отклонение понижает эффективность контакта в несколько раз. 

В этой области уже производились исследования \cite{net_tech}\cite{fiber_net_sys}, но при поиске среди доступных источников не было найдено метода, для расчета нашего случая, то есть стыковки оптического волокна с полосковым волноводом. В работе разбирается данный случай, проводится моделирование происходящих при стыковке процессов, показывается эффективность оптических переходов в зависимости от точности юстировки. 