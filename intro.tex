\section*{Введение}
Использование оптических технологий и элементов является большим прогрессом в наше время. Меньшая длина волны позволяет достичь большей частоты сигналов, в частности при передаче данных. Такая частота дает возможность передавать потоки информации в несколько терабит в секунду. Важными преимуществами ВОЛС являются такие факторы, как малое затухание сигналов, позволяющее, при использовании современных технологий, строить участки оптических систем в сто и более километров без ретрансляции, высокая помехозащищенность, связанная с малой восприимчивостью оптического волокна к электромагнитным помехам, и многие другие.

Развитие оптической связи создало новый класс устройств - интергрально-оптические схемы и компоненты. Их устройство похоже на обычные интегральные платы и представляет собой представляет собой подложку из электрооптического кристалла и выполненных в ней канальных волноводов, которые могут служить базой для изготовления различных функциональных элементов  (поляризаторов, делителей, модуляторов и др.). Основным достоинством интегрально-оптических устройств является их высокое быстродействие. Уже созданы интегрально-оптические  переключатели с временем переключения менее 100 фс. Такое быстродействие недостижимо для устройств обычной полупроводниковой электроники. Возможность передачи и обработки больших объемов информации определяет бурное развитие интегральной оптики в настоящее время.

Область исследований постепенно расширяется и сейчас она включает в себя все исследования, направленные на использование волноводной технологии для создания новых или усовершенствования существующих оптических приборов. Разработаны компактные и миниатюрные элементы, чей малый размер обеспечивает большую надежность, лучшую механическую и температурную стабильность, уменьшение потребления энергии и управляющих напряжений в активных приборах. Кроме того, отдельные волноводные элементы могут объединяться вместе в более сложныхе схемы на общей подложке или на отдельных платах. Эти новые волноводные приборы, например, лазеры и модуляторы, могут успешно конкурировать по свои индивидуальным качествам с их объемными оптическим эквивалентами.

Элементы интегрально-оптических схем схожи с элементами электронных интегральных схем и при их изготовлении используются аналогичные технологии. Это обстоятельство и обусловило возникновение самого названия новой области - "интегральная оптика". В действительности исследования по интегральной оптике были начаты около 40 лет назад. Появление лазеров стимулировало эти исследования, и были достигнуты значительные успехи. Сейчас созданы все элементарные компоненты интегрально-оптических схем: планарные повлновода, эффективные устройства ввода-вывода излучения в волновод, пленочные переключатели, ответвители, модуляторы, источнники излучения и фотодетекторы, а также линзы, призмы, отражатели и поляризаторы в тонкопленочном исполнении.

Несмотря на достигнутые успехи, по-прежнему остается ряд проблем, затрудняющих развитие этой области. Одна из них - это проблема соединения компонентов. Интегрально-оптические схемы представляют собой подложку со сформированным на ней полосковым волноводом, и соединяются они при помощи оптического волокна. На этих соединениях возникают потери в связи с различиями в волноводном  распространении света в полосковом и цилиндрическом волноводах. Кроме того, необходима точная юстировка компонентов и их стабильное положение, даже небольшое отклонение понижает эффективность контакта в несколько раз. 

В данной работе разбирается эта проблема и проводится моделирование происходящих при этом процессов, и показывается эффективность стыков в зависимости от точности юстировки.