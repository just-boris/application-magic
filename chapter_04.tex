\chapter{Технико-экономический анализ расчета эффективности связи двух волноводов}
\section{Выбор аналога объекта разработки}
Аналог объекта разработки — это объект, имеющий аналогичное функциональное назначение и являющийся лучшим по своим технико-эксплуатационным характеристикам на данный момент времени.
Проведенный патентный и литературный, а также поиск в сети Интернет показали, что программные средства, обеспечивающие расчет эффективности связи мод двух волноводов, являются закрытыми.

\section{Определение товарного типа объекта разработки}
Товарный тип объекта разработки устанавливается путем анализа рыночной цели его создания. С этой точки зрения выделяются следующие типы:
\begin{itemize}
	\item разработки, выполняемые с коммерческой целью, то есть предназначенные для реализации на рынке. Общей характерной чертой  таких разработок является более или менее широкий спрос на их результаты на рынке (наличие нескольких потребителей). Такие разработки могут быть двух типов: имеющие рыночный аналог и не имеющие  рыночного аналога. Аналогом объекта разработки считается объект, имеющий аналогичное функциональное назначение и являющийся лучшим по своим технико-эксплуатационным характеристикам на данный момент времени.
	\item разработки, выполняемые с некоммерческой целью, то есть не предназначенные для прямой или косвенной реализации на рынке.
\end{itemize}
Целью данной дипломной работы является разработка программы, для расчета эффективности связи мод двух волноводов. Эта разработка относится к разработкам, выполняемым с некоммерческой целью, то есть не предназначенным для прямой или косвенной реализации на рынке (научные исследования фундаментального и поискового характера, новые разработки, выполняемые для конкретного заказчика, в частности, единственного потребителя). 
Согласно классификации объект разработки относится к товарному типу V (разработки, выполняемые с некоммерческой целью) \cite{economics}. Отнесение объекта разработки к определенному товарному типу служит основанием для определения состава расчетов в экономической части. В нашем случае в экономический расчет входит только смета затрат на разработку по методу сметного калькулирования.

\section{Расчет сметы затрат на разработку}
В состав сметной стоимости разработки входят следующие статьи затрат:
\begin{itemize}
\item материалы, покупные изделия и полуфабрикаты;
\item специальное оборудование для проведения разработки;
\item основная заработная плата разработчиков;
\item дополнительная заработная плата;
\item отчисления в социальные внебюджетные фонды;
\item затраты на электроэнергию для технологических целей;
\item затраты на командировки;
\item контрагентские работы;
\item прочие затраты;
\item накладные расходы.
\end{itemize}
Сметная стоимость определяется методом сметного калькулирования. Метод сметного калькулирования основан на прямом определении затрат по отдельным статьям.

\subsection{Расчет затрат на комплектующие изделия и полуфабрикаты}
Для создания данного продукта не использовались материалы и покупные изделия, поскольку он является результатом умственного труда, поэтому затраты на комплектующие отсутствуют
\subsection{Расчет амортизационных отчислений}
При использовании наличного оборудования в смету включаются только амортизационные отчисления. Амортизационные отчисления наличного оборудования рассчитывается по формуле \ref{amort}:
\begin{equation}
C_\textit{об} = \sum_{j=1}^m H_A \cdot \textit{Ц}_\textit{об} \cdot t_n \mbox{, (руб.)}
\label{amort}
\end{equation}
где  $H_A$ – годовая норма амортизационных отчислений;\\
$\textit{Ц}_\textit{об}$ – цена j-ого вида оборудования, руб;\\
$t_n$ – время использования оборудования для работы, лет.

Годовая норма амортизационных отчислений $H_A$ определяется по формуле \ref{amort_year}:
\begin{equation}
H_A = \frac{1}{T_H}\mbox{, (руб.)}
\label{amort_year}
\end{equation}
где   $T_H$ - нормативный срок службы оборудования, лет. 

В целях снижения затрат использовалось только свободное программное обеспечение. Специальное оборудование для проведения разработки и затраты на него представлено в таблице \ref{amort_items}
\begin{table}[h]
	\caption{Амортизационные отчисления}
	\label{amort_items}
	\begin{tabular}{|l|l|c|c|c|c|c|c|c|}
		\hline
			\No & \thead{Наименование\\оборудования} & \thead{Ед.\\ изм.} & К-во & \thead{Цена\\единицы,\\руб.} & \thead{Время\\ использования,\\ лет} & \thead{Нормативный\\ срок службы,\\ лет} & $N_A$ & \thead{Сумма,\\ руб}\\
		\hline
			1 & ЭВМ & шт. & 2 & 25000 & 2 & 10 & 0,10 & 10000,00 \\
		\hline
	\end{tabular}
\end{table}

\subsection{Расчет затрат на основную заработную плату}
Основная заработная плата сотрудников ($C_oc$) определяется по формуле \ref{zarplat} 
\begin{equation}
	C_{oc} = \sum_{j=1}^k \textit{П}_{mj} \cdot \overline{\textit{З}_{mj}} \cdot P \mbox{, (руб.)}
	\label{zarplat}
\end{equation}
где  $k$ – количество категорий разработчиков;\\
$\textit{П}_{mj}$ – количество разработчиков данной категории;\\
$\overline{\textit{З}_{mj}}$ - среднечасовая заработная плата j-категории разработчиков, руб/час;\\
$P$ – продолжительность работы, выполняемой работником определенной категории, час.

В разработке и реализации данного проекта участвовали 2 две категории разработчиков: руководитель и один инженер-разработчик.

Среднечасовая заработная плата руководителя рассчитывается с учетом того, что его среднемесячная заработная плата составляет 40\,000 рублей. Если рабочий день принять восьмичасовым, а количество рабочих дней в месяце – 22, то его среднечасовая заработная плата составит:
$$
	\textit{З}_{m1} = \frac{40\,000}{8 \cdot 22} = 227.27 \mbox{ руб/час}
$$
Среднечасовая заработная плата инженера-разработчика рассчитывается с учетом того, что его среднемесячная заработная плата составляет 35\,000 рублей. Если рабочий день принять восьмичасовым, а количество рабочих дней в месяце – 22, то его среднечасовая заработная плата составит:
$$
	\textit{З}_{m1} = \frac{35\,000}{8 \cdot 22} = 198.86 \mbox{ руб/час}
$$
Общее время затраченное на разработку программного обеспечения составит:
$$
	Р = 4*22*8 = 704 час.
$$
Основная заработная плата работников рассчитана в таблице \ref{zp_table}:

\begin{table}[h]
	\caption{Основная заработная плата работников}
	\label{zp_table}
	\begin{tabular}{|l|l|l|l|}
		\hline
			Сотрудник & \thead{Среднечасовая\\ заработная плата,\\руб./час} & \thead{Время\\работы,\\час} & \thead{Основная\\заработная плата,\\руб.}\\
		\hline
			Руководитель & 227.27 & 704 & 159\,998 \\
		\hline
			Инженер & 198.86 & 704 & 139\,997 \\
		\hline			
			Итого: & & & 299\,995 \\
		\hline					
	\end{tabular}
\end{table}

\subsection{Расчет дополнительной заработной платы сотрудников}
Дополнительная заработная плата сотрудников, проводящих разработку ($C_\textit{доп}$) определяется по формуле \ref{zarplat_ext}:
\begin{equation}
	C_\textit{доп} = \frac{C_{oc} \cdot d}{100} \mbox{, (руб.)}
	\label{zarplat_ext}
\end{equation}  

где $d$ – норматив затрат на дополнительную зарплату от основной, $d=10\%$.
$$
	C_\textit{доп}  = 299\,995 \cdot 0.10 = 29\,999.5 \mbox{ руб.}
$$

\subsection{Расчет затрат на отчисления в социальные внебюджетные фонды}
Отчисления в социальные внебюджетные фонды определяются по формуле \ref{zarplat_fonds}

\begin{equation}
	C_\textit{сф} = \frac{(C_{oc} + C_\textit{доп}) \cdot r}{100} \mbox{, (руб.)}
	\label{zarplat_fonds}
\end{equation} 
где $r$ – суммарная величина единого социального налога  и отчисления на страхование от несчастных случаев. По состоянию на 01.01.2013: $r = 22.0\%	+ 2.9\% + 5.1\% = 30\%$.
$$
	C_\textit{сф} = (299\,995 + 29\,999.5)\cdot 0.3 = 98\,998 \mbox{ руб.}
$$

\subsection{Расчет затрат на технологическое топливо и энергию}
Затраты на электроэнергию для технологических целей определяются по формуле \ref{electricity}:
\begin{equation}
	C_\textit{эн} = \sum_{i=1}^l W_i \cdot T_i \cdot C_{kr} \cdot K_{wi} \mbox{, (руб.)}
	\label{electricity}
\end{equation}

где  $l$ – номеклатура оборудования, используемого для разработки;\\
$W_i$ – мощность оборудования по паспорту, кВт;\\
$T_i$ – время использования для проведения разработки, час;\\
$C_{kr}$ – стоимость одного кВт\textperiodcentered час электроэнергии, руб;\\
$K_{wi}$ – коэффициент использования мощности ($K_{wi} < 1$).\\

\begin{table}[h]
	\caption{Затраты на электроэнергию для технологических целей}
	\begin{tabular}{|l|l|l|l|l|l|}
		\hline
			Оборудование & \thead{Мощность\\ по паспорту,\\кВт} & \thead{Время\\использования,\\час} & \thead{Коэффициент\\использования} & \thead{Стоимость\\одного\\кВт-час, руб} &  \thead{Затраты,\\руб} \\
		\hline
			ЭВМ & 0.4 & 704 & 0.95 & 3 & 802.56 \\
		\hline
			Итого: & & & & & 802.56 \\
		\hline		
	\end{tabular}
\end{table}

\subsection{Расчет затрат на командировки}
Затраты на командировки не производились.

\subsection{Расчет затрат на контрагентские работы}
Контрагентские работы ($C_\textit{кр}$) не проводились

\subsection{Расчет прочих затрат}
К статье <<прочие затраты>> ($C_\textit{п}$) относятся затраты, связанные с оплатой экспертиз, консультаций, получения патентной информации, арендой помещения, и затраты на канцелярские товары и т.д. Эти затраты задаются в процентах к суммарной величине предыдущих статей ($10\%$) по формуле \ref{cost_other}
\begin{equation}
	C_\textit{п} = (C_\textit{м}+C_\textit{об}+C_\textit{ос}+C_\textit{доп}+C_\textit{сф}+ C_\textit{эн}+C_\textit{кр})\cdot r \mbox{, (руб.)}
	\label{cost_other}
\end{equation}
где $r=10\%$.
$$
	C_\textit{П} = (10\,000 + 299\,995 + 98\,998 + 802.56)\cdot 0.1 = 40\,979.5 \mbox{ руб.}
$$

\subsection{Расчет накладных расходов}
Накладные расходы ($C_\textit{н}$) определяются по формуле \ref{cost_util}, где $r = 80\%$:
\begin{equation}
	C_\textit{н} = C_\textit{ос} \cdot r \mbox{, (руб.)}
	\label{cost_util}
\end{equation}
$$
	C_\textit{н} = 299\,995 \cdot 0.8 = 239\,996 \mbox{ руб.}
$$

\subsection{Расчет общей сметной стоимости}
Общая сметная стоимость ($C_\textit{р}$) определяется суммированием ее составляющих:
\begin{equation}
	C_\textit{р} = C_\textit{м}+C_\textit{об}+C_\textit{ос}+C_\textit{доп}+C_\textit{сф}+ C_\textit{эн}+C_\textit{кр}+C_\textit{п}+C_\textit{н} \mbox{, (руб.)}
\end{equation}

\begin{table}[h!]
	\caption{Сметная стоимость проведения разработки}
	\begin{tabular}{|c|p{9cm}|c|c|}
		\hline
			\thead{\No \\ п/п} & Статьи расходов & \thead{Условные\\обозначения} & \thead{Затраты по\\ статьям, руб.}\\
		\hline
			1 & Стоимость специального оборудования & $C_\textit{об}$ & 10\,000 \\
		\hline
			2 & Основная заработная плата & $C_\textit{ос}$ & 299\,995 \\
		\hline
			3 & Дополнительная заработная плата & $C_\textit{доп}$ & 29\,999.5 \\
		\hline
			4 & Отчисления в социальные внебюджетные фонды & $C_\textit{сф}$ & 98\,998 \\
		\hline
			5 & Затраты на электроэнергию для технологических целей  & $C_\textit{сф}$ & 802.56 \\
		\hline
			6 & Прочие прямые затраты & $C_\textit{п}$ & 40\,979.5 \\
		\hline		
			7 & Накладные расходы & $C_\textit{н}$ & 239\,996 \\
		\hline		
			\multicolumn{2}{|l|}{Общая сметная стоимость} & $C_\textit{р}$ & 720\,770.56 \\
		\hline
	\end{tabular}
\end{table}